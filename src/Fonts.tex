\ExplSyntaxOn

% \RequirePackage{fontspec} % in unicode-math

\RequirePackage[
    math-style=ISO, % Macht gr. Großbuchstaben kursiv
    bold-style=ISO 
    ]{unicode-math} 
% \ifJHPreamble@PackagesByPath
    % \newcommand{\__FontPath}{fonts/opentype}
% \else
%     \newcommand\__FontPath{}
% \fi
    
\newcommand{\__FontPath}{fonts/opentype/}

\ifJHPreamble@StandardMathFonts
\else
    \ifJHPreamble@KpMath
        %%% kpfonts haben einige Zeichen nicht, deshalb zu STIX2Math gewechselt. Passen beide nicht wirklich zu MinionPro

        % \RequirePackage{kpfonts}
        % \RequirePackage[mathcal,bbsans,notext]{kpfonts-otf}
        \setmathfont[
            Path=\__FontPath,
            version=normal
        ]{KpMath-Regular.otf}

        \setmathfont[
            Path=\__FontPath,
            version=bold
        ]{KpMath-Bold.otf}

        % % \setmathfont{KpMath-Semibold.otf}[Path=\__FontPath,version=semibold]
        % \setmathfont{KpMath-Sans.otf}[Path=\__FontPath,version=sans]
        % \setmathfont{KpMath-Light.otf}[Path=\__FontPath,version=light]


        % \newfontfamily{\fallbackfont}{STIX2Math}[Path=\__FontPath,
        %     Extension = .otf
        % ]

        % \DeclareTextFontCommand{\usefallbackfont}{\fallbackfont}

    \else
        \setmathfont[
            Path=\__FontPath,
            version=bold
        ]{XITSMath-Bold.otf}

        %%% STIX2Math und XITS (STIX)
        \ifJHPreamble@STIX
            \setmathfont[
                Path=\__FontPath,
                version=normal
            ]{STIX2Math.otf}
        \else
            \setmathfont[
                Path=\__FontPath,
                version=normal
            ]{XITSMath.otf} 
        \fi
    \fi


    % \setmathfont[
    %     Path=\__FontPath,
    %     version=sans
    % ]{GFSNeohellenicMath.otf}
    
    % \setmathfont[
    %     Path=\__FontPath,
    %     range={"2983-"29FC},
    %     version=sans
    % ]{XITSMath.otf}
    
    \setmathfont[
        Path=\__FontPath,
        % range={up, it, bb, frak, cal, scr, sfup, sfit, bfup,bfit, bfcal},
        version=sans
    ]{KpMath-Sans.otf}
    % \setmathfont[
    %     Path=\__FontPath,
    %     range={it,up, bb},
    %     version=sans
    % ]{FiraMath-Regular.otf}
    %%%



    \mathversion{normal}
\fi

%%% Text Fonts

\ifJHPreamble@StandardTextFonts
\else
    % \IfFontExistsTF{MinionPro}{
    \setmainfont{MinionPro}[
        Path=\__FontPath,
        Extension = .otf,
        UprightFont = *-Medium,
        BoldFont = *-Bold, %SemiBold
        ItalicFont = *-It, %
        BoldItalicFont = *-BoldIt, %SemiboldIt
        Language = German
    ] % Minion Pro und Myriad Pro sind anscheinend zur privaten Verwendung legal (aus der Installation von Adobe Acrobat Reader zu bekommen)
    % }{}

    %   \IfFontExistsTF{Myriad-Pro}{
    \setsansfont{Myriad-Pro}[ %SansSerif 
        Path=\__FontPath, 
        UprightFont = *-Regular.ttf,
        BoldFont = *-Bold.ttf, %SemiBold
        ItalicFont = *-It.otf, %
        Language = German,
        % Script = Default, % Seems to be on CustomDefault although that doesn't exist in the documentaion
        Contextuals = ResetAll
    ]
    % }{}



    %%% Braucht kompliziert konvertierte und installierte Myriad und Minion Type 1 Schriften mit Paketen. (Für pdflatex)

    % \RequirePackage{alphabeta}
    % \RequirePackage[Path=\__FontPath,lf]{MinionPro}
    % \RequirePackage[Path=\__FontPath,
    % 	scale=1,
    % 	lf,
    % 	sansmath]{MyriadPro}

\fi

\newfontfamily\codeFont{CascadiaCode}[
    Path=\__FontPath, 
    UprightFont = *-Regular.ttf,
    BoldFont = *-Bold.ttf
]

%%%% FallbackFont
% Mit Kp-Math notwendig, weil die das nicht hatten.
% Diese Kommandos könnte man wahrscheinlich auch durch \setmathfont{STIXMath2}[Unicode-range] irgendwie weglassen.

% \RequirePackage{newunicodechar}
% \newunicodechar{Δ}{\increment} % Könnte man machen, aber unnötig. Es gibt das Zeichen ∆, das ist dasselbe wie \increment, aber verschieden von Δ (\Delta)

% \newcommand{\MakeCharAlwaysUseFont}[2][\fallbackfont]{\newunicodechar{#2}{#1{#2}}}
% \newcommand{\MakeCharAlwaysUseFontAndTextMode}[2][\fallbackfont]{\newunicodechar{#2}{\text{#1{#2}}}}

% \NewDocumentCommand\MakeCharactersAlwaysUseFallbackFont{ > { \SplitList {,} } m }{ \ProcessList{#1}{\MakeCharAlwaysUseFont} }
% \NewDocumentCommand\MakeCharactersAlwaysUseFallbackFontAndTextMode{ > { \SplitList {,} } m }{ \ProcessList{#1}{\MakeCharAlwaysUseFontAndTextMode} }


% \MakeCharactersAlwaysUseFallbackFontAndTextMode{𝟙}

% \MakeCharactersAlwaysUseFallbackFontAndTextMode{⥎,⥋,⥊,⥐}