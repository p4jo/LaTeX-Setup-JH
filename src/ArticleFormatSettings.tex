\addtokomafont{sectioning}{\rmfamily\boldmath\color{dunkelblau}}
%\addtokomafont{sectioning}{\mathversion{sansbold}}
\addtokomafont{pagehead}{\normalfont\color{dunkelgrau}}
\addtokomafont{caption}{\sffamily\mathversion{sans}\footnotesize}
\addtokomafont{captionlabel}{\sffamily\bfseries\footnotesize}
\setcapindent{0cm}
%\addtokomafont{pagenumber}{\sffamily}
\setkomafont{pagenumber}{\normalfont}

\renewcaptionname{ngerman}{\figurename}{Abb.}
\renewcaptionname{ngerman}{\figureautorefname}{Abb.}
\renewcaptionname{ngerman}{\tableautorefname}{Tabelle} %War auf 'Abschnitt' ?


% \linespread{1.3}

% \renewcommand{\topfraction}{0.85}
% \renewcommand{\bottomfraction}{0.85}
% \renewcommand{\textfraction}{0.1}
% \renewcommand{\floatpagefraction}{0.8}


\clubpenalty9998
\widowpenalty9999
\displaywidowpenalty=9999


\usepackage{amsthm}
\usepackage{thmtools}

\declaretheoremstyle[
    bodyfont=\upshape,
    headfont={\sffamily\bfseries},
    notefont={\sffamily\bfseries},
    headpunct={:},
]{JHDefinition}

\declaretheoremstyle[
    bodyfont={\color{dunkelgrau}}
]{JHProof}

\declaretheoremstyle[
    bodyfont={\upshape%\color{dunkelgrau}
        },
    headfont=\itshape,
    notefont=\itshape,
    headpunct={:},
]{JHRemark}


\declaretheorem[
    style=JHDefinition,
    shaded={
        bgcolor=blasshellblau
    }
]{definition}
\declaretheorem[
    style=JHDefinition,
    shaded={
        bgcolor=hellgrün
    }
]{Satz}
\declaretheorem[
    style=JHDefinition,
    shaded={
        bgcolor=hellgrau
    },
    numberlike=Satz
]{Lemma}
\declaretheorem[
    style=JHDefinition,
    shaded={
        bgcolor=hellgrau
    },
    numberlike=Satz
]{Korollar}
\declaretheorem[
    style=JHDefinition,
    numbered=no
]{Beispiel}

\declaretheorem[
    style=JHProof,
    numbered=no
]{Beweis}

\declaretheorem[
    style=JHRemark,
    name=Bemerkung,
    numbered=no
]{remark}
\declaretheorem[
    style=JHRemark,
    name=Bemerkung,
    numbered=no
]{bem}
\declaretheorem[
    style=JHRemark,
    name=Bemerkung,
    numbered=no
]{Bemerkung}
