
\usepackage{amsthm}
\usepackage{thmtools}

\declaretheoremstyle[
    bodyfont=\upshape,
    headfont={\sffamily\bfseries},
    notefont={\sffamily\bfseries},
    headpunct={:},
]{JHDefinition}

\declaretheoremstyle[
    bodyfont={\color{dunkelgrau}},
    headpunct={:},
    headfont={\itshape},
    notefont={\itshape}
]{JHProof}

% Bemerkung mit Rahmen (man muss die Umgebung noch in leftLined setzen)
\declaretheoremstyle[
    bodyfont={\upshape},
    headpunct={:},
    headfont={\itshape}, %\itshape
    notefont={\itshape}
]{JHRemark}

\cs_if_exist:NTF\chapter{
    \newcommand\__numberwithin{chapter}
}{
    \newcommand\__numberwithin{section}
}
\newcommand\theorem_language_set_ngerman:{
    \str_set:cx{ theorem_name_theorem } { Satz }
    \str_set:cx{ theorem_name_lemma } { Lemma }
    \str_set:cx{ theorem_name_corollary } { Korollar }
    \str_set:cx{ theorem_name_definition } { Definition }
    \str_set:cx{ theorem_name_defLemma } { Definierendes~Lemma }
    \str_set:cx{ theorem_refname_defLemma } { Definierenden~Lemma }
    \str_set:cx{ theorem_name_proof } { Beweis }
    \str_set:cx{ theorem_name_question } { Frage }
    \str_set:cx{ theorem_name_remark } { Bemerkung }
    \str_set:cx{ theorem_name_example } { Beispiel }
}
\newcommand\theorem_language_set_english:{
    \str_set:cx{ theorem_name_theorem } { Theorem }
    \str_set:cx{ theorem_name_lemma } { Lemma }
    \str_set:cx{ theorem_name_corollary } { Corollary }
    \str_set:cx{ theorem_name_definition } { Definition }
    \str_set:cx{ theorem_name_defLemma } { Defining~Lemma }
    \str_set:cx{ theorem_refname_defLemma } { Defining~Lemma }
    \str_set:cx{ theorem_name_proof } { Proof }
    \str_set:cx{ theorem_name_question } { Question }
    \str_set:cx{ theorem_name_remark } { Remark }
    \str_set:cx{ theorem_name_example } { Example }
}
\cs_if_exist:NTF\captionsenglish{
    \addto\captionsenglish{\theorem_language_set_english:}
}{
    \theorem_language_set_english:
}
\cs_if_exist:NT\captionsngerman{
    \addto\captionsngerman{\theorem_language_set_ngerman:}
}

\declaretheorem[
    style=JHDefinition,
    shaded={
        rulecolor=white, % xcolor throws error 'undefined color shaderulecolor', when omitExternalPackages, even though Packages1, Packages2, BibliographySettings and GeneralSetup don't seem to do anything regarding this.
        bgcolor=hellgrün
    },
    numberwithin=\__numberwithin,
    name=\protect\theorem_name_theorem
]{Satz}
\declaretheorem[
    style=JHDefinition,
    shaded={
        rulecolor=white,
        bgcolor=hellgrün
    },
    numberlike=Satz,
    name=\protect\theorem_name_lemma
]{Lemma}
\declaretheorem[
    style=JHDefinition,
    shaded={
        rulecolor=white,
        bgcolor=hellgrau
    },
    numberlike=Satz,
    name=\protect\theorem_name_corollary
]{Korollar}

% numbered like Definition
\ifJHPreamble@NumberDefLikeSatz
    \declaretheorem[
        style=JHDefinition,
        shaded={
            rulecolor=white,
            bgcolor=blasshellblau
        },
        numberlike=Satz,
        name=\protect\theorem_name_definition
    ]{Definition}
\else
    \declaretheorem[
        style=JHDefinition,
        shaded={
            rulecolor=white,
            bgcolor=blasshellblau
        },
        numberwithin=\__numberwithin,
        name=\protect\theorem_name_definition
    ]{Definition}
\fi
\declaretheorem[
    style=JHDefinition,
    shaded={
        rulecolor=white,
        bgcolor=blasstürkis
    },
    name=\protect\theorem_name_defLemma,
    refname=\protect\theorem_refname_defLemma,
    numberlike=Definition
]{DefLemma}

% not numbered
% \declaretheorem[
%     style=JHRemark,
%     numbered=no,
%     name=Beispiel
% ]{beispiel}
\ifJHPreamble@removeProofs
    \NewDocumentEnvironment{Beweis}{o +b}{}{} % b collects the body of the environment :) 
\else
    \declaretheorem[
        style=JHProof,
        numbered=no,
        name=\protect\theorem_name_proof
    ]{Beweis}
\fi

\declaretheorem[
    style=JHDefinition,
    shaded={
        rulecolor=white,
        bgcolor=Orange
    },
    numbered=no,
    name=\protect\theorem_name_question
]{Frage}
% \declaretheorem[
%     style=JHRemark,
%     name=Bemerkung,
%     numbered=no
% ]{bemerkung}

% \usepackage{mdframed}
% \surroundwithmdframed[
%     linewidth=1pt,
%     linecolor= dunkelgrau,
%     % roundcorner=1pt,
%     topline = false,
%     rightline = false,
%     bottomline = false,
%     rightmargin=0pt,
%     skipabove=0.1cm,
%     skipbelow=0pt,
%     leftmargin=1pt,
%     innerleftmargin=10pt,
%     innerrightmargin=0pt,
%     innertopmargin=-20pt,
%     innerbottommargin=0pt,
% ]{Bemerkung}

\usepackage[most]{tcolorbox}
\tcbuselibrary{skins}
\tcbuselibrary{breakable}
\newcommand\drawLeftLine[2][-8pt]{ % #2=the color
    \draw[line~width = 2pt, #2, opacity = 0.7]
        ([ shift = { (1pt, #1) } ] interior.north~west )
    -- ([ shift = { (1pt, 8pt) } ] interior.south~west );
}
\newtcolorbox{leftLineBox}[2][]{
    enhanced,
    empty,
    breakable=true,
    left=7pt,
    overlay={ \drawLeftLine[-25pt]{#2} },
    % code for the first part of a break sequence:
    skin~first~is~subskin~of = { emptyfirst }{ 
        overlay = {\drawLeftLine[-25pt]{#2} }
    },
    % code for the middle part of a break sequence:
    skin~middle~is~subskin~of = { emptymiddle }{
        overlay = { \drawLeftLine{#2} }
    },
    % code for the last part of a break sequence:
    skin~last~is~subskin~of = { emptylast }{
        overlay = { \drawLeftLine{#2} }
    },
    #1
}

\NewDocumentEnvironment{leftLined}{m D(){dunkelgrün}}{ 
    \begin{leftLineBox}{#2}
            \hspace { -11pt }
            \itshape 
            #1 :
            \upshape
}{ 
    \end{leftLineBox}
}

\NewDocumentCommand{DeclareLinedTheorem}{o m m}{ 
    \declaretheorem[#1]{#2}
    \NewDocumentEnvironment{#2}{o D(){#3}}{
        \begin{leftLined}{
            #2 \IfValueT { ##1 } { ~ ( ##1 ) } 
        }(##2)
    }{
        \end{leftLined}
    }

}

\NewDocumentEnvironment{Bemerkung}{o D(){dunkelgrün}}{
    \begin{leftLined}{
        \theorem_name_remark \IfValueT { #1 } { ~ ( #1 ) } 
    }(#2)
}{
    \end{leftLined}
}
\NewDocumentEnvironment{Beispiel}{o D(){dunkelblau}}{
    \begin{leftLined}{
         \theorem_name_example \IfValueT { #1 } { ~ ( #1 ) } 
    }(#2)
}{
    \end{leftLined}
}