% xspace
\xspaceaddexceptions{“ "-}

% interval
% \intervalconfig{soft open fences} %kann man verwenden, um Intervalle (a,b] zu erzeugen
% \intervalconfig{separator symbol=;} %kann man verwenden, um Intervalle ]a;b] zu erzeugen

% siunitx
\sisetup
{
	detect-all, %Alle äußeren Font-Veränderungen werden übernommen
	locale=DE, %insb. , statt .
	range-phrase={\ bis\ }, %in \SIrange
	separate-uncertainty=true, % (1,32 \pm 0, 22) statt 1,32(22) in Ausgabe
	%table-figures-uncertainty = 1, % reserves space for one \pm sign (use as \tabular{S[tab..]})
	% EITHER
		% per-mode=symbol-or-fraction,
	% OR
		per-mode=fraction, %
		fraction-function=\sfrac,
	% ENDEITHER
	% round-integer-to-decimal=true,
	% quotient-mode=fraction, % Option wurde entfernt
	% literal-superscript-as-power=true % Option nicht mehr notwendig
}

\shorthandoff{"} % Würde normalerweise "a zu ä machen.

% hyperref
\hypersetup
{
	colorlinks	= true,
	linkcolor   = dunkelblau, 
	citecolor   = dunkelblau,
	bookmarksnumbered = true,
	urlcolor    = dunkelblau,
	pdfauthor   = {\JHPreamble@author},
	pdftitle    = {\JHPreamble@title}%,
	% pdfsubject  = {<set this yourself using \hypersetup>},
	% pdfkeywords = {<same>}
}

% lstlisting

\lstset{
    basicstyle=\scriptsize\codeFont,
    keywordstyle=\bfseries\color{lila},
    emphstyle=\color{dunkelrot},
    stringstyle=\color{dunkelgrün},
    commentstyle=\color{dunkelgrau},
    identifierstyle=\color{dunkelblau},
    numbers=left,
    stepnumber=5,
    frame=tb,
    showstringspaces=false,
    showspaces=false,
    showtabs=false,
    breaklines=true,
    tabsize=2
}

\lstdefinelanguage
    [JH]{Python} % dialect JH of Python
    []{Python} % based on standard dialect of Python
{
    morekeywords={self,True,False,as,assert},           
    emph={__init__,__str__,__repr__,\#\#\#\#\#\#\#}
}

% tensind
% \tensordelimiter{§} % Funktioniert nicht, da unicode-math so normale Zeichen überschreibt, auf eine Art und Weise, die tensind nicht überschreibt
\tensordelimiter{■}
\tensorformat{}

% \hyphenation{Hoch-tem-pera-tur-su-pra-lei-tern Ku-pfer-atomen}


\usepackage{amsthm}
\usepackage{thmtools}

\declaretheoremstyle[
    bodyfont=\upshape,
    headfont={\sffamily\bfseries},
    notefont={\sffamily\bfseries},
    headpunct={:},
]{JHDefinition}

\declaretheoremstyle[
    bodyfont={\color{dunkelgrau}}
]{JHProof}

\declaretheoremstyle[
    bodyfont={\upshape%\color{dunkelgrau}
        },
    headfont=\itshape,
    notefont=\itshape,
    headpunct={:},
]{JHRemark}


\declaretheorem[
    style=JHDefinition,
    shaded={
        bgcolor=blasshellblau
    },
    numberwithin=chapter
]{definition}
\declaretheorem[
    style=JHDefinition,
    shaded={
        bgcolor=hellgrün
    },
    numberwithin=chapter
]{Satz}
\declaretheorem[
    style=JHDefinition,
    shaded={
        bgcolor=hellgrün
    },
    numberlike=Satz
]{Lemma}
\declaretheorem[
    style=JHDefinition,
    shaded={
        bgcolor=blasstürkis
    },
    name=Definierendes\ Lemma,
    numberlike=Satz
]{DefLemma}
\declaretheorem[
    style=JHDefinition,
    shaded={
        bgcolor=hellgrau
    },
    numberlike=Satz
]{Korollar}
\declaretheorem[
    style=JHDefinition,
    numbered=no,
    name=Beispiel
]{Beispiel}

\declaretheorem[
    style=JHProof,
    numbered=no
]{Beweis}

\declaretheorem[
    style=JHRemark,
    name=Bemerkung,
    numbered=no
]{remark}
\declaretheorem[
    style=JHRemark,
    name=Bemerkung,
    numbered=no
]{bem}
\declaretheorem[
    style=JHRemark,
    name=Bemerkung,
    numbered=no
]{Bemerkung}
