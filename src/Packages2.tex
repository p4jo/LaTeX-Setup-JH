% After unicode-math and fonts
%     \RequirePackage{microtype} % sorgt für besseres Schriftbild, indem es micro-Anpassungen macht. 
\RequirePackage[autostyle=true]{csquotes} % \enquote{...} statt "...". Lokalisierbare Anführungszeichen 

\RequirePackage{graphicx} % für \Grafik
\RequirePackage{animate} % für gif-\Grafik 

\RequirePackage{pdfpages} %Um pdfs anzuhängen 
\RequirePackage[export]{adjustbox} %Um Bilder nur eingeschränkt anzuhängen

\RequirePackage{listings} % Um code anzuhängen. Einstellungen in GeneralSetup


\ifJHPreamble@omitFloat
\else
    \RequirePackage{float} % Um \begin{figure}[H] zu machen
\fi

\RequirePackage{color}
% \RequirePackage[dvipsnames]{xcolor}

\RequirePackage{amsmath} % in kpfonts

% \RequirePackage{breqn} % Verändert equation parsing, sodass sie automatische Zeilenumbrüche haben.

\RequirePackage{siunitx} %\si{\kilo\metre\per\second\squared} \num{3,14(7)E-4} \SI{8,7+-2.3}\ampere\
% Einstellungen in GeneralSetup

% \RequirePackage{ziffer} %macht Kommas in Zahlen zu ,\! oder so, aber besser wäre \num aus siunitx


\RequirePackage{physics} %\bra, \ket, \qty (macht Klammern zu \left \right Paaren)

\RequirePackage{interval} % um Intervalle gut zu setzen. Einstellungen in format.

% \RequirePackage{titling} % Um Titelseite zu verändern, müsste man aber mit der KOMA-Script Klasse eh können

% \RequirePackage{pgffor} % Funktionalität wahrscheinlich in etoolbox oder lua oder xparse (-> LaTeX3) enthalten

\RequirePackage{cancel} % Durchstreichen in Mathe

\RequirePackage{xspace} % Ein Leerzeichen, das man ans Ende von Befehlen bauen kann, das keines wird, wenn da ein Satzzeichen steht
\RequirePackage{scalerel} % Größe von Symbolen anpassen (für Große Operatoren à la \sum)
\RequirePackage{xfrac} % für \sfrac, ein schräger Bruch mit kleinerer Schrift. Leider etwas seltsam

% \RequirePackage{amssymb} % geht nicht gescheit. Geht nicht zusammen mit unicode-math % Inhalt eh in kpfonts-otf drin

% \RequirePackage{stmaryrd} % Eigentlich nur für \llbracket, das gibts in kpfonts (als \lBrack) und in unicode-math
% \RequirePackage{bm} % braucht es auch mit kpfonts nicht
%\RequirePackage{upgreek} % braucht's mit unicode-math natürlich auch nicht

\RequirePackage{tensind} 